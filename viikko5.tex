\documentclass[12pt,a4paper]{article}
% !TEX program = xelatex
\usepackage[utf8]{inputenc}
\usepackage[T1]{fontenc}
\usepackage[finnish]{babel}
\usepackage[utf8]{inputenc}
\usepackage{graphicx}
\usepackage{titling}
\usepackage{titlesec}
\usepackage{booktabs}
\usepackage{fancyhdr}
\usepackage{lipsum}
\usepackage{comment}
\usepackage{enumitem}
\usepackage{xcolor}
\usepackage{longtable}
%\usepackage{cite}
\usepackage{pgfgantt}
\usepackage{amsmath, amssymb}
\usepackage{tikz}
\usepackage[margin=1in]{geometry}
\usepackage[backend=biber, style=numeric]{biblatex}
%\usepackage{hyperref}
\usepackage{bookmark}
\usepackage{enumitem}
\usepackage{amsmath}
\usepackage{listings}
\lstset{language=Python, basicstyle=\ttfamily\small, breaklines=true,columns=fullflexible}
\lstset{escapeinside={(*@}{@*)}}
\usepackage{fontspec}
\setmainfont{Arial}
\newfontfamily\stylishfont{Noteworthy}
%\newfontfamily\stylishfont{Zapfino}
%\addbibresource{references.bib}
\usetikzlibrary{calc}
\usepackage{xcolor}

\lstdefinestyle{pidstyle}{
    basicstyle=\ttfamily\footnotesize,
    breaklines=true,
    escapechar=\#, % Define escape character for inline LaTeX commands
    linewidth=\textwidth,
    basicstyle=\ttfamily\scriptsize
}

\renewcommand{\maketitle}{%
  \begin{leftmark}
    \vspace*{\baselineskip} % Add a bit of vertical space

%    \includegraphics[width=4cm]{example-image-a} % Add an image before the title. you will need to replace the image path with your own

%    \vspace{0.5cm} % Add vertical space before title

    \textbf{\fontsize{18}{36}\selectfont \thetitle} % Font Size and Bold Title

     \vspace{0.05cm} % Add vertical space before subtitle
%    \textit{\Large \theauthor}  % Subtitle / Author
    \vspace{\baselineskip} % Add vertical space after subtitle
     \rule{\textwidth}{0.4pt} % Add a horizontal line

   \end{leftmark}
%    \thispagestyle{empty} % Prevent header/footer on the title page
}


% Section Formatting
\titleformat{\section}
  {\normalfont\fontsize{18}{22}\bfseries} % Font and style
  {\thesection}         % Section number
  {1em}                   % Horizontal space after section number
  {}                     % Code before the section name
  []                     % Code after the section name

\titleformat{\subsection}
  {\normalfont\fontsize{14}{18}\bfseries} % Font and style
  {\thesubsection}         % Subsection number
  {1em}                   % Horizontal space after subsection number
  {}                     % Code before the subsection name
  []                     % Code after the subsection name

\setlength{\parindent}{0pt}

\title{Computing platforms (Spring 2025)\newline
week 6}
\author{Juha-Pekka Heikkilä}



\pagestyle{fancy}
\fancyhf{}

\renewcommand{\headrulewidth}{0pt}

\newcommand{\footerline}{\makebox[\textwidth]{\hrulefill}}

\newcommand{\footercontent}{%
    \begin{tabular}{@{}l@{}}
        \footerline \\
        \leftmark \hfill \rlap{\thepage}
    \end{tabular}
}

\fancyfoot[C]{\footercontent}


\newcommand{\exercise}[1]{
    \section*{Tehtävä #1}
    \markboth{Tehtävä #1}{}
}

\addtolength{\hoffset}{-1.75cm}
\addtolength{\textwidth}{3.5cm}
%\addtolength{\voffset}{-3cm}
%\addtolength{\textheight}{6cm}
%\setlength{\parindent}{0pt}



% (a), (b), (c)
\newlist{kohta}{enumerate}{1}
\setlist[kohta,1]{
  label=\textbf{\makebox[1cm][l]{\Huge\text{(\stylishfont\alph*)}}},
  leftmargin=!,
  labelindent=0pt
}

% (1), (2), (3)
\newlist{alakohta}{enumerate}{1}
\setlist[alakohta,1]{
  label=\textbf{\makebox[1cm][l]{\Large\text{(\arabic*)}}},
  leftmargin=!,
  labelindent=0pt
}

% termi: selitys
\newlist{kuvaus}{description}{1}
\setlist[kuvaus]{%
  font=\bfseries\stylishfont,
  labelsep=0.5cm,
  leftmargin=2.5cm,
  style=nextline
}

\newcommand{\korostus}[2][yellow]{\colorbox{#1}{\strut #2}}
%\korostus{Yksi kirjoittaja on jo sisällä}
%\korostus[red]{Lukijan täytyy odottaa jos kirjoittajia on paikalla}
%\korostus[orange]{Tämä osa ei ole suoritettavissa}


\newcommand{\evalslantti}[4][-12]{%
%  \left. #2 \,\right|% ei indeksejä tähän
  \mkern-10mu\raisebox{0pt}[0pt][0pt]{\rotatebox{#1}{$\Big|$}}% vinoviiva päälle
  \mkern3mu{}_{\!#3}^{\!#4}% arvot viivan oikealle puolelle
}



\newcommand{\evalraise}{1.2ex}
\newcommand{\evallow}{1.2ex}

% vino eval-viiva, arvot oikealla (oletus: -12)
% \evalslant[asteet]{lauseke}{ala}{yla}
\newcommand{\evalslant}[4][-12]{%
  \left. #2 \,\right.%
  \mkern-10mu\raisebox{0pt}[0pt][0pt]{\rotatebox{#1}{$\Big|$}}%
  \mkern2mu{}^{\raisebox{\evalraise}{$\scriptstyle #4$}}_{\raisebox{-\evallow}{$\scriptstyle #3$}}%
}

\title{MAT12003 Todennäköisyyslaskenta I — Viikko 5}
\date{}

\begin{document}

\maketitle

\exercise{1}
Olkoon $X$ jatkuva satunnaismuuttuja, jonka tiheysfunktio on 
$$f(x)=\begin{cases}
    \frac{3a^3}{x^{4}},\text{ kun }x\ge a,\\
    0 \text{ muulloin,}
\end{cases}$$
missä $a>0$ on reaalinen parametri. Laske Var$(X)$.

\vspace{0.4cm}


Odotusarvot:
\[
E(X)=\int_a^\infty x\,f(x)\,dx
=3a^3\int_a^\infty x^{-3}\,dx
=3a^3\,\evalslantpre{-\frac{1}{2}x^{-2}}{a}{\infty}
=\frac{3}{2}\,a
\]
\[
E(X^2)=\int_a^\infty x^2\,f(x)\,dx
=3a^3\int_a^\infty x^{-2}\,dx
=3a^3\,\evalslantpre{-x^{-1}}{a}{\infty}
=3a^2
\]
\vspace{0.4cm}

Varianssi (sijoitetaan suoraan vaan määritelmään 6.8):
\[
\operatorname{Var}(X)=E(X^2)-\big(E(X)\big)^2
=3a^2-\Big(\tfrac{3}{2}a\Big)^2
=\frac{3}{4}\,a^2
\]






\pagebreak
\exercise{2}
Olkoot $X$, $Y$ ja $Z$
riippumattomia satunnaismuuttujia, joilla kaikilla on sama odotusarvo $\mu$ ja
varianssi $\sigma^2$. Laske odotusarvo ja varianssi
satunnaismuuttujille (a) $X+2Y+3Z$, (b) $X-Y$, (c) $XY$.

\begin{kohta}

\item $X+2Y+3Z$
\[
\begin{aligned}
    E(X+2Y+3Z) &=\mu+2\mu+3\mu=6\mu \\
    \Var(X+2Y+3Z) &=\sigma^2+4\sigma^2+9\sigma^2=14\sigma^2
\end{aligned}
\]

\item $X-Y$
\[
\begin{aligned}    
E(X-Y)&=\mu-\mu=0 \\
\text{käytetään Huomautus 6.22 ja Lause 6.23:}\\
\Var(X-Y)&=\sigma^2+\sigma^2=2\sigma^2
\end{aligned}
\]

\item $XY$ \\
 (Esimerkki 6.25 ii)
\[
E(XY)=E(X)E(Y)=\mu^2
\]
Sitten: \\
$E(X^2)=\Var(X)+[E(X)]^2=\sigma^2+\mu^2$\\
ja: \\
 $E(Y^2)=\sigma^2+\mu^2$\\
joten
\[
\begin{aligned}
\Var(XY)&=E(X^2Y^2)-[E(XY)]^2\\
&=E(X^2)E(Y^2)-\mu^4\\
&=(\sigma^2+\mu^2)^2-\mu^4\\
&=\sigma^4+2\sigma^2\mu^2
\end{aligned}
\]

\end{kohta}





\pagebreak
\exercise{3}
Tasoon piirretään kolmio, jonka kärkinä ovat origo
sekä $x$- ja $y$-akselilta satunnaisesti valitut pisteet $X$ ja $Y$,
jotka ovat riippumattomia ja noudattavat normaalijakaumaa parametrein
$(0, 1)$. Laske kolmion pinta-alan odotusarvo.
\vspace{0.4cm}

eli kolmio millä kärjet (0,0), (X,0) ja (0,Y), missä X\sim N(0,1) ja 
Y\sim N(0,1) riippumattomia ja kolmion pinta-ala on
\[
A=\frac{|X||Y|}{2}
\]
Riippumattomuudesta seuraa E[|X|\,|Y|]=E[|X|]\,E[|Y|] ja koska X ja Y on
samanlaisia merkitään E[|X|]\,=\,E[|Y|] ja 
E[|X|]\,=\,E[|Y|]\,=\,E[|Z|] kun Z\sim N(0,1) jolloin voidaan kirjoittaa


\[
E[A]=\frac{E[|X|]E[|Y|]}2
=\frac{\bigl(E[|Z|]\bigr)^2}2
\]
ja lasketaan \(E[|Z|]\) (Luento09\_handout lause 5 todistusta apuna käyttäen)
\[
\begin{aligned}
E[|Z|]
&= 2\int_{0}^{\infty} x\;\frac{1}{\sqrt{2\pi}}\,e^\frac{-x^{2}}2\,dx
= \frac{2}{\sqrt{2\pi}}\int_{0}^{\infty} x\,e^\frac{-x^{2}}2\,dx\\[4pt]
&= \frac{2}{\sqrt{2\pi}}\;\evalslantpre{-e^\frac{-x^{2}}2}{0}{\infty}
= \frac{2}{\sqrt{2\pi}}\,(0-(-1))
= \sqrt{\frac{2}{\pi}}
\end{aligned}
\]

Siis
\[
E[A]
=\frac{\bigl(E[|Z|]\bigr)^2}2
=\frac{\left(\sqrt{\frac{2}{\pi}}\right)^{2}}2
=\frac{1}{\pi}\approx 0{,}31831
\]





\end{document}