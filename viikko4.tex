\documentclass[12pt,a4paper]{article}
\input{yhteiset-asetukset.tex}

\title{MAT12003 Todennäköisyyslaskenta I — Viikko 4}
\date{}

\begin{document}

\maketitle

\exercise{1}
Heitetään tavallista noppaa ja nostetaan korttipakasta kortti. Olkoon $X$ suurempi nopan silmäluvusta ja kortin arvosta (ässän arvo on 1, jätkän 11, kuningattaren 12 ja kuninkaan 13). Määritä satunnaismuuttujan $X$ odotusarvo.

%Heitetään tavallista noppaa ja nostetaan korttipakasta kortti. Olkoon $X$ suurempi nopan silmäluvusta ja kortin arvosta (ässän arvo 1, jätkän 11, kuningattaren 12 ja kuninkaan 13). Määritä satunnaismuuttujan $X$ odotusarvo.
\vspace{0.4cm}
Olkoon $X=\max(D,K)$, missä
noppa $D\in\{1,\dots,6\}$ ja kortti $K\in\{1,\dots,13\}$.
Heitot on riippumattomat\\


PTNF. Riippumattomuuden vuoksi:
\[
F_X(x)=P(X\le x)=P(D\le x,\;K\le x)=P(D\le x)\,P(K\le x).
\]
jaetaan tapaukset
\[
F_X(x)=
\begin{cases}
\displaystyle \frac{x}{6}\cdot\frac{x}{13}=\frac{x^2}{78} & x=1,2,\dots,6,\\[6pt]
\displaystyle 1\cdot\frac{x}{13}=\frac{x}{13} & x=7,8,\dots,13,
\end{cases}
\quad\text{ja }F_X(0)=0
\]
pistetodennäköisyysfunktio (ptnf) saadaan erotuksena:
\[
P(X=x)=F_X(x)-F_X(x-1)=
\begin{cases}
\displaystyle \frac{x^2-(x-1)^2}{78}=\frac{2x-1}{78} & x=1,2,\dots,6\\[8pt]
\displaystyle \frac{x-(x-1)}{13}=\frac{1}{13} & x=7,8,\dots,13
\end{cases}
\]

\textbf{Odotusarvo} 
\[
\mathbb{E}[X]=\sum_{x=1}^{13} x\,P(X=x)
=\sum_{x=1}^{6} x\cdot\frac{2x-1}{78}+\sum_{x=7}^{13} x\cdot\frac{1}{13}
\]
Lasketaan summat:
\[
\sum_{x=1}^{6} x(2x-1)=2\sum_{x=1}^{6}x^2-\sum_{x=1}^{6}x
=2\cdot 91-21=161
\]
\[
\sum_{x=7}^{13} x = \frac{(7+13)\cdot 7}{2}=70
\]
Siis
\[
\mathbb{E}[X]=\frac{161}{78}+\frac{70}{13}
=\frac{161}{78}+\frac{420}{78}
=\frac{581}{78}\approx 7{,}449
\]

%$F_X(x)=P(\max(D,K)\le x)$ tarkoittaa, että molempien on oltava korkeintaan $x$\\
%Kun $x\le6$, saadaan $P(D\le x)=x/6$ ja $P(K\le x)=x/13$ $\Rightarrow F_X(x)=x^2/78$ \\
%Kun $x\ge7$, noppa on varmasti $\le x$,
%joten $F_X(x)=x/13$ \\
%Pistetodennäköisyys on peräkkäisten $F_X$-arvojen erotus.

\end{document}