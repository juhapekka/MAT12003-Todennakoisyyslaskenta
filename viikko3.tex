\documentclass[12pt,a4paper]{article}
\input{yhteiset-asetukset.tex}
\newcommand{\riip}{\perp \!\!\! \perp}

\title{MAT12003 Todennäköisyyslaskenta I — Viikko 3}
\date{}

\begin{document}

\maketitle

\exercise{1}
Heitetään kahta noppaa. Olkoon $A =$ ”ainakin toinen silmäluvuista on parillinen” ja $B =$ ”silmälukujen summa on vähintään 9”. Ovatko $A$ ja $B$ riippumattomia?


\vspace{0.4cm}


Perusjoukko $|\Omega|=36$.

\[
P(A)=1-\Big(\tfrac36\Big)^2=\tfrac{27}{36}=\tfrac34,\qquad
P(B)=\tfrac{10}{36}=\tfrac{5}{18}
\]
summaa $\ge 9$ on 10 tapausta ja niistä vain $(5,5)$ ei kuulu $A$:han

{\small
\[
\begin{array}{c|l|c|c}
\text{Summa} & \text{tulokset} & \# & \text{väh. yksi parillinen}\\
\hline
9  & (3,6),(4,5),(5,4),(6,3) & 4 & 4 \\
10 & (4,6),(5,5),(6,4)       & 3 & 2 \\
11 & (5,6),(6,5)             & 2 & 2 \\
12 & (6,6)                   & 1 & 1 \\
\hline
   & \text{yhteensä}         & 10& 9
\end{array}
\]}

joten
\[
P(A\cap B)=\tfrac{9}{36}=\tfrac14
\]
Koska
\[
P(A)P(B)=\tfrac34\cdot\tfrac{5}{18}=\tfrac{5}{24}\ \ne\ \tfrac14=P(A\cap B)
\]
\vspace{0.4cm}

tapahtumat $A$ ja $B$ eivät ole riippumattomia.







\pagebreak
\exercise{2}
Korttipakasta nostetaan 13 korttia (ilman takaisinpanoa). Olkoon $X$ ässien lukumäärä otoksessa. Laske tapahtumien $\{X=3\}$ ja $\{X\ge 2\}$ todennäköisyydet. Mitä jakaumaa $X$ noudattaa?
\vspace{0.4cm}

Koska nostot tehdään ilman takaisinpanoa populaatiosta
(N=52) jossa on K=4 ässää, ässien lukumäärä X noudattaa
hypergeometrista jakaumaa: $X\sim\mathrm{Hyperg}(52,4,13)$
\vspace{0.4cm}

\[
P(X=k)=\frac{\binom{4}{k}\,\binom{48}{13-k}}
{\binom{52}{13}},\qquad k=0,1,2,3,4
\]

\begin{kohta}
\item $P(X=3)$:
\[
P(X=3)=\frac{\binom{4}{3}\binom{48}{10}}{\binom{52}{13}}
=\frac{858}{20825}\approx 0{,}04120
\]

\item $P(X\ge 2)$ (komplementtina):
%\[
%f(k\ge 2) = P(X\ge 2)=1-P(X\le 1)
%=1-\frac{\binom{48}{13}+\binom{4}{1}\binom{48}{12}}{\binom{52}{13}}
%=\frac{5359}{20825}\approx 0{,}25733
%\]

\[ P(X \ge 2) = 1 - P(X \le 1) = 1 - \big( P(X=0) + P(X=1) \big) \]
\[ 
\begin{aligned}
&= 1 - \left( \frac{\binom{4}{0}\binom{48}{13}}{\binom{52}{13}} + \frac{\binom{4}{1}\binom{48}{12}}{\binom{52}{13}} \right)\\
&=\frac{5359}{20825}\approx 0{,}25733
\end{aligned}
\]
\end{kohta}





\pagebreak
\exercise{3}
Olkoot $A$, $B$ ja $C$ tapahtumia, joille $P(A)=P(B)=0{,}25$ ja $P(C)=0{,}3$. Lisäksi tiedetään, että $A\riip B$ ja $B\riip C$ sekä $A$ ja $C$ ovat erillisiä. Laske $P(A\cup B\cup C)$.\\

\noindent Vinkki: Muista, että $A\cup B\cup C=(A\cup B)\cup C$ ja $(A\cup B)\cap C=(A\cap B)\cup(B\cap C)$.


\begin{alakohta}
\item Hajotus ja leikkauksen lasku:
\[
A\cup B\cup C=(A\cup B)\cup C,\qquad
(A\cup B)\cap C=(A\cap C)\cup(B\cap C)
\]
Koska $A$ ja $C$ ovat erillisiä, $A\cap C=\varnothing$, joten
\[
(A\cup B)\cap C = B\cap C
\]

\item $P(A\cup B)$ riippumattomuudesta:
\[
P(A\cup B)=P(A)+P(B)-P(A\cap B),
\quad A\riip B\Rightarrow P(A\cap B)=P(A)P(B)=\tfrac14\cdot\tfrac14=\tfrac{1}{16}
\]
Siis:
\[
P(A\cup B)=\tfrac14+\tfrac14-\tfrac{1}{16}=\tfrac{7}{16}
\]

\item $P(B\cap C)$ riippumattomuudesta:
\[
B\riip C\Rightarrow P(B\cap C)=P(B)P(C)=\tfrac14\cdot\tfrac{3}{10}=\tfrac{3}{40}
\]

\item Edelliset yhdessä:
\[
\begin{aligned}
P(A\cup B\cup C)
&=P(A\cup B)+P(C)-P\big((A\cup B)\cap C\big)\\
&=\tfrac{7}{16}+\tfrac{3}{10}-\tfrac{3}{40}
= \frac{35+24-6}{80}
= \frac{53}{80}
=0{,}6625
\end{aligned}
\]
\end{alakohta}






\end{document}