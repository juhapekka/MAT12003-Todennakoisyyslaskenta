\documentclass[12pt,a4paper]{article}
\input{yhteiset-asetukset.tex}
\newcommand{\riip}{\perp \!\!\! \perp}

\title{MAT12003 Todennäköisyyslaskenta I — Viikko 3}
\date{}

\begin{document}

\maketitle

\exercise{1}
Heitetään kahta noppaa. Olkoon $A =$ ”ainakin toinen silmäluvuista on parillinen” ja $B =$ ”silmälukujen summa on vähintään 9”. Ovatko $A$ ja $B$ riippumattomia?


\vspace{0.4cm}


Perusjoukko $|\Omega|=36$.

\[
P(A)=1-\Big(\tfrac36\Big)^2=\tfrac{27}{36}=\tfrac34,\qquad
P(B)=\tfrac{10}{36}=\tfrac{5}{18}
\]
summaa $\ge 9$ on 10 tapausta ja niistä vain $(5,5)$ ei kuulu $A$:han

{\small
\[
\begin{array}{c|l|c|c}
\text{Summa} & \text{tulokset} & \# & \text{väh. yksi parillinen}\\
\hline
9  & (3,6),(4,5),(5,4),(6,3) & 4 & 4 \\
10 & (4,6),(5,5),(6,4)       & 3 & 2 \\
11 & (5,6),(6,5)             & 2 & 2 \\
12 & (6,6)                   & 1 & 1 \\
\hline
   & \text{yhteensä}         & 10& 9
\end{array}
\]}

joten
\[
P(A\cap B)=\tfrac{9}{36}=\tfrac14
\]
Koska
\[
P(A)P(B)=\tfrac34\cdot\tfrac{5}{18}=\tfrac{5}{24}\ \ne\ \tfrac14=P(A\cap B)
\]
\vspace{0.4cm}

tapahtumat $A$ ja $B$ eivät ole riippumattomia.







\pagebreak
\exercise{2}
Korttipakasta nostetaan 13 korttia (ilman takaisinpanoa). Olkoon $X$ ässien lukumäärä otoksessa. Laske tapahtumien $\{X=3\}$ ja $\{X\ge 2\}$ todennäköisyydet. Mitä jakaumaa $X$ noudattaa?
\vspace{0.4cm}

Koska nostot tehdään ilman takaisinpanoa populaatiosta
(N=52) jossa on K=4 ässää, ässien lukumäärä X noudattaa
hypergeometrista jakaumaa: $X\sim\mathrm{Hyperg}(52,4,13)$
\vspace{0.4cm}

\[
P(X=k)=\frac{\binom{4}{k}\,\binom{48}{13-k}}
{\binom{52}{13}},\qquad k=0,1,2,3,4
\]

\begin{kohta}
\item $P(X=3)$:
\[
P(X=3)=\frac{\binom{4}{3}\binom{48}{10}}{\binom{52}{13}}
=\frac{858}{20825}\approx 0{,}04120
\]

\item $P(X\ge 2)$ (komplementtina):
%\[
%f(k\ge 2) = P(X\ge 2)=1-P(X\le 1)
%=1-\frac{\binom{48}{13}+\binom{4}{1}\binom{48}{12}}{\binom{52}{13}}
%=\frac{5359}{20825}\approx 0{,}25733
%\]

\[ P(X \ge 2) = 1 - P(X \le 1) = 1 - \big( P(X=0) + P(X=1) \big) \]
\[ 
\begin{aligned}
&= 1 - \left( \frac{\binom{4}{0}\binom{48}{13}}{\binom{52}{13}} + \frac{\binom{4}{1}\binom{48}{12}}{\binom{52}{13}} \right)\\
&=\frac{5359}{20825}\approx 0{,}25733
\end{aligned}
\]
\end{kohta}





\pagebreak
\exercise{3}
Olkoot $A$, $B$ ja $C$ tapahtumia, joille $P(A)=P(B)=0{,}25$ ja $P(C)=0{,}3$. Lisäksi tiedetään, että $A\riip B$ ja $B\riip C$ sekä $A$ ja $C$ ovat erillisiä. Laske $P(A\cup B\cup C)$.\\

\noindent Vinkki: Muista, että $A\cup B\cup C=(A\cup B)\cup C$ ja $(A\cup B)\cap C=(A\cap B)\cup(B\cap C)$.


\begin{alakohta}
\item Hajotus ja leikkauksen lasku:
\[
A\cup B\cup C=(A\cup B)\cup C,\qquad
(A\cup B)\cap C=(A\cap C)\cup(B\cap C)
\]
Koska $A$ ja $C$ ovat erillisiä, $A\cap C=\varnothing$, joten
\[
(A\cup B)\cap C = B\cap C
\]

\item $P(A\cup B)$ riippumattomuudesta:
\[
P(A\cup B)=P(A)+P(B)-P(A\cap B),
\quad A\riip B\Rightarrow P(A\cap B)=P(A)P(B)=\tfrac14\cdot\tfrac14=\tfrac{1}{16}
\]
Siis:
\[
P(A\cup B)=\tfrac14+\tfrac14-\tfrac{1}{16}=\tfrac{7}{16}
\]

\item $P(B\cap C)$ riippumattomuudesta:
\[
B\riip C\Rightarrow P(B\cap C)=P(B)P(C)=\tfrac14\cdot\tfrac{3}{10}=\tfrac{3}{40}
\]

\item Edelliset yhdessä:
\[
\begin{aligned}
P(A\cup B\cup C)
&=P(A\cup B)+P(C)-P\big((A\cup B)\cap C\big)\\
&=\tfrac{7}{16}+\tfrac{3}{10}-\tfrac{3}{40}
= \frac{35+24-6}{80}
= \frac{53}{80}
=0{,}6625
\end{aligned}
\]
\end{alakohta}







\pagebreak
\exercise{4}
Heitetään painotettua kolikkoa kolme kertaa. Heitot ovat riippumattomia toisistaan, ja kruunan todennäköisyys on $p$, ($0<p<1$). Osoita, että tapahtumat
\begin{align*}
A&=\text{''saadaan sekä kruunia että klaavoja''},\\
B&=\text{''saadaan enintään yksi klaava''}
\end{align*}
ovat riippumattomia, jos ja vain jos $p=\frac{1}{2}$.


\begin{alakohta}
\item Perustodennäköisyydet:
\[
P(A)=1-\big(p^3+(1-p)^3\big)=3p(1-p)
\]
(A: vähintään yksi kumpaakin \Leftrightarrow \, ei ole
kolme kruunaa eikä kolme klaavaa)

\[
P(B)=P(\text{0 klaavaa})+P(\text{1 klaava})
= p^3+3p^2(1-p)=p^2(3-2p)
\]

\[
P(A\cap B)=P(\text{täyttyy $A$ ja $\le1$ klaavaa})
= P(\text{tasan 1 klaava})
=3p^2(1-p)
\]

\item Riippumattomuusehto:\\ $A\riip B \iff P(A\cap B)=P(A)P(B)$
\[
3p^2(1-p)\;=\; \big(3p(1-p)\big)\,\big(p^2(3-2p)\big)
\quad\Longleftrightarrow\quad
1=p(3-2p)
\]
koska $0<p<1$ \Rightarrow \, voidaan jakaa tekijällä $3p^2(1-p)>0$\\
Tästä
\[
3p-2p^2=1\ \Longleftrightarrow\ 2p^2-3p+1=0
\ \Longleftrightarrow\ (2p-1)(p-1)=0
\]
Sallitulla välillä ratkaisu on \(p=\tfrac12\)

\item Takaisinpäin ("vain jos"): \\Kun \(p=\tfrac12\):
\[
P(A)=3\cdot\tfrac12\cdot\tfrac12=\tfrac34,\quad
P(B)=\tfrac14(3-1)=\tfrac12,\quad
P(A\cap B)=3\cdot\tfrac14\cdot\tfrac12=\tfrac38=\tfrac34\cdot\tfrac12
\]
Siis \(P(A\cap B)=P(A)P(B)\), joten \(A\riip B\)
\end{alakohta}

\noindent Siis: \(A\) ja \(B\) ovat riippumattomia jos ja 
vain jos \(p=\tfrac12\)










\pagebreak
\exercise{5}
Heitetään ensin tavallista 6-sivuista noppaa, sitten 
4-sivuista noppaa (jossa silmäluvut 1, 2, 3, 4 ovat yhtä todennäköiset).
Olkoon heittojen tulokset $X$ ja $Y$, ja $S=|X-Y|$. Laske satunnaismuuttujan $S$ pistetodennäköisyydet ja piirrä pistetodennäköisyysfunktiota havainnollistava pylväsdiagrammi sekä kertymäfunktion kuvaaja.\\




Kaikkia pareja on $6\cdot 4=24$. S saa arvot $s\in\{0,1,2,3,4,5\}$\\

\textbf{Pistetodennäköisyysfunktio (ptnf):}
Merkitään $p_S(s)=P(S=s)$
\[
p_S(s)=\frac{\#\{(x,y):|x-y|=s\}}{24}
\]

\emph{miksi nämä lukumäärät:} 
ehto $|x-y|=s$ tarkoittaa joko $x-y=s$ tai $y-x=s$:

{\small
\[
\begin{array}{c|c|c|c|c}
s & \#\{(x,y):x-y=s\} & \#\{(x,y):y-x=s\} & \text{yht.} & p_S(s)=\dfrac{\#}{24} \\\hline
0 & \#\{(1,1),(2,2),(3,3),(4,4)\}=4 & \text{--} & 4 & \dfrac{4}{24} \\
1 & 4 & 3 & 7 & \dfrac{7}{24} \\
2 & 4 & 2 & 6 & \dfrac{6}{24} \\
3 & 3 & 1 & 4 & \dfrac{4}{24} \\
4 & 2 & 0 & 2 & \dfrac{2}{24} \\
5 & 1 & 0 & 1 & \dfrac{1}{24} \\
\end{array}
\]
}
\begin{comment}
   
\begin{itemize}[leftmargin=1.2cm]
  \item $s=0$: parit $(1,1),(2,2),(3,3),(4,4)$ $\Rightarrow$ $\#=4$.
  \item $s=1$: $x-y=1$ tuottaa $x=2,3,4,5$ (4 kpl), ja $y-x=1$ tuottaa $y=2,3,4$ (3 kpl) $\Rightarrow$ $\#=7$.
  \item $s=2$: vastaavasti $4+2=6$.
  \item $s=3$: $3+1=4$.
  \item $s=4$: $2+0=2$.
  \item $s=5$: $1+0=1$.
\end{itemize}
\end{comment}

Siis ptnf on
\[
\begin{aligned}
p_S(0)&=\tfrac{4}{24}=\tfrac{1}{6},&
p_S(1)&=\tfrac{7}{24},&
p_S(2)&=\tfrac{6}{24}=\tfrac{1}{4},\\
p_S(3)&=\tfrac{4}{24}=\tfrac{1}{6},&
p_S(4)&=\tfrac{2}{24}=\tfrac{1}{12},&
p_S(5)&=\tfrac{1}{24}.
\end{aligned}
\]

\textbf{Kertymäfunktio:}
\[
F_S(s)=P(S\le s),\quad \text{diskreetit pisteet:}\quad
\begin{array}{c|cccccc}
s & 0 & 1 & 2 & 3 & 4 & 5\\\hline
F_S(s) & \tfrac{1}{6} & \tfrac{11}{24} & \tfrac{17}{24} & \tfrac{21}{24} & \tfrac{23}{24} & 1
\end{array}
\]
\pagebreak

\textbf{Kuvaajat:}


\begin{figure}[h]
  \centering
  \includegraphics[width=.6\textwidth]{viikko3tehtävä5.jpg}
%  \caption{ptnf $p_S(s)=P(S=s)$}
\end{figure}






\pagebreak
\exercise{6}
Lentoyhtiö tietää kokemuksesta, että keskimäärin 5 \% paikan varanneista jää saapumatta
koneeseen. Siksi yhtiö myykin 257 lippua koneeseen, johon mahtuu 250 matkustajaa. Millä
todennäköisyydellä jokainen koneeseen saapuva saa paikan?\\



\begin{alakohta}
  \item Y on poisjääneiden lukumäärä. Silloin:
  \[
    Y \sim \mathrm{Bin}(n=257,\ p=0{,}05),\qquad
    P(Y=k)=\binom{257}{k}(0{,}05)^k(0{,}95)^{257-k}
  \]
  Saapuvien määrä on A=N-Y

  \item Kaikki mahtuvat $\iff$ saapuvia korkeintaan C:
  \[
    A \le C\ \Longleftrightarrow\ N - Y \le C\ \Longleftrightarrow\ Y \ge N-C = 257-250=7
  \]
  Siis:
  \[
    P(\text{kaikki mahtuvat koneeseen})=P(Y\ge 7)
  \]

  \item Kirjoitus binomin avulla. Käytetään komplementtia:
  \[
    P(Y\ge 7)
    = \sum_{k=7}^{257}\binom{257}{k}(0{,}05)^k(0{,}95)^{257-k}
    = 1 - \sum_{k=0}^{6}\binom{257}{k}(0{,}05)^k(0{,}95)^{257-k}
  \]

  \item Numeroarvo:\\

   Pythonilla:
   {\small
   \begin{verbatim}
      >>> import math
      >>> n = 257
      >>> p = 0.05
      >>> print(1 - sum(math.comb(n, k) * (p**k) * ((1 - p)**(n - k)) for k in range(7)))
      0.9745257525592507
   \end{verbatim}
   }

   Siis:
  \[
    P(Y\ge 7)\ \approx\ 0{,}97453 \quad (\text{noin }97{,}45\%)
  \]


\end{alakohta}






\pagebreak
\exercise{7}
Hajamielisellä herra H:lla on nipussaan $n$ avainta, 
joista yksi sopii hänen oveensa.
Herra H ei kuitenkaan muista, mikä. Olkoon $X$ sen kerran 
järjestysluku, jolla ovi aukeaa. Laske $X$:n 
pistetodennäköisyysfunktio ja kertymäfunktio olettaen, että herra
H valitsee avaimen umpimähkään ja
\begin{enumerate}
\item[(a)] muistaa mitä avaimia on jo kokeillut,
\item[(b)] ei muista mitä avaimia hän on kokeillut.
\end{enumerate}
Mitä jakaumaa $X$ noudattaa kohdassa (a)? Entä kohdassa (b)?
\vspace{0.4cm}




\begin{kohta}
\item \textbf{muistaa mitä avaimia on jo kokeillut, (ei siis yritä samaa kahdesti)}

Idea on että yritykset on "ilman takaisinpanoa" ja avaimet
käydään läpi satunnaisessa järjestyksessä.\\

\textbf{Pistetodennäköisyysfunktio (ptnf):}
\[
p_X(k)=P(X=k)=\frac{1}{n},\qquad k=1,2,\dots,n
\]

\textbf{Kertymäfunktio:}
\[
F_X(k)=P(X\le k)=\frac{k}{n},\qquad k=1,2,\dots,n
\]



X on \textbf{diskreetti tasajakauma}
%\pagebreak
\item \textbf{ei muista mitä avaimia hän on kokeillut. (voi siis yrittää samaa uudelleen)}

Ideo on että jokaisella yrityksellä valitaan satunnaisesti yksi n:stä avaimesta, 
riippumattomasti edellisistä. Onnistumistodennäköisyys jokaisella
yrityksellä on 1/n\\

\textbf{Pistetodennäköisyysfunktio (ptnf)}:
\[
p_X(k)=P(X=k)=\Big(1-\frac{1}{n}\Big)^{k-1}\,\frac{1}{n},\qquad k=1,2,\dots
\]

\textbf{Kertymäfunktio:}
\[
F_X(k)=P(X\le k)=1-\Big(1-\frac{1}{n}\Big)^{k},\qquad k=1,2,\dots
\]


X on \textbf{geometrinen jakauma}
\end{kohta}







\pagebreak
\exercise{8}
Laske todennäköisyys, että $X$ on parillinen, jos
\begin{enumerate}
\item[(b)] $X\sim\operatorname{Geom}(p)$.
\item[(a)] $X\sim\operatorname{Poisson}(\lambda)$,
\end{enumerate}
Vinkki: Kaavasta $e^x=\sum_{k=0}^\infty\frac{x^k}{k!}$ voisi olla hyötyä (b)-kohdassa.\\
\vspace{0.2cm}

\korostus{Nää tehtävät alla on samassa järjestyksessä kuin tehtäväpaperissa, ensin b sitten a.}
\vspace{0.2cm}


\begin{kohta}
\item[\textbf{(b)}] \textbf{$X\sim\mathrm{Geom}(p)$}, ptnf $P(X=k)=p(1-p)^k,\ k=0,1,2,\dots$
\[
\begin{aligned}
P(X\ \text{parillinen})
&=\sum_{m=0}^\infty p(1-p)^{2m}
= p\sum_{m=0}^\infty \big((1-p)^2\big)^m
= \frac{p}{1-(1-p)^2}
= \frac{1}{\,2-p\,}
\end{aligned}
\]

\item[\textbf{(a)}] \textbf{$X\sim\mathrm{Poisson}(\lambda)$}, ptnf: $P(X=k)=e^{-\lambda}\dfrac{\lambda^k}{k!} ,\qquad k=0,1,2,\dots$
\[
\begin{aligned}
P(X\ \text{parillinen})
&=\sum_{m=0}^\infty e^{-\lambda}\frac{\lambda^{2m}}{(2m)!}
= e^{-\lambda}\sum_{m=0}^\infty\frac{\lambda^{2m}}{(2m)!}
\end{aligned}
\]
Käytetään vihjettä $e^x=\sum_{k\ge0}\frac{x^k}{k!}$ sekä $e^{-x}=\sum_{k\ge0}\frac{(-1)^k x^k}{k!}$\\
yhteenlaskemalla:
\[
e^{\lambda}+e^{-\lambda}
=\sum_{k=0}^\infty \frac{(1+(-1)^k)\lambda^k}{k!}
=2\sum_{m=0}^\infty \frac{\lambda^{2m}}{(2m)!}
\]
siksi:
\[
\sum_{m=0}^\infty \frac{\lambda^{2m}}{(2m)!}
=\frac{e^{\lambda}+e^{-\lambda}}{2}
\]
siis:
\[
P(X\ \text{parillinen})
= e^{-\lambda}\cdot\frac{e^{\lambda}+e^{-\lambda}}{2}
= \frac{1+e^{-2\lambda}}{2}
\]
\end{kohta}






\pagebreak
\exercise{9}
Tutkitaan ensin kahden 6-sivuisen nopan heittoa, tulokset
satunnaismuuttujia $X$ ja $Y$, ja niiden summa $S=X+Y$. Moodlesta löytyvä
R-koodi \texttt{noppasumma.r} käy läpi kaikki parin $(X,Y)$ mahdolliset
arvot, laskee kussakin tapauksessa summan $S$, ja laskee yhteen kutakin
summan arvoa vastaavat todennäköisyydet. Toisin sanoen koodi laskee $S$:n
pistetodennäköisyydet tarkasti (laskentatarkkuuden rajoissa toki). Huomaa,
että kyseessä ei ole satunnaiskokeesta saatu epätarkka mittaustulos, vaan 
koko perusjoukko on käyty tarkasti läpi.

Kokeile koodia, ja muokkaa sitä sitten niin, että se laskee summan $S$
pistetodennäköisyydet, kun heittää kolmea noppaa tuloksi $X$, $Y$, $Z$
ja tutkitaan summan $S=X+Y+Z$ jakaumaa. Liitä vastaukseen muokkaamasi 
koodi sekä laskemasi pistetodennäköisyydet. Jos haluat, voit piirtää
ne esim.\ komennolla \texttt{plot(f, type="h")}.\\



\textbf{Kahden ja kolmen nopan summan ptnf}

\paragraph{Alkuperäinen kahden nopan koodi (\texttt{noppasumma.r}).}
\begin{lstlisting}[language=R]
# Luodaan vektori f, jonka jokainen komponentti on 0
f = rep(0,12);

# K�yd��n l�pi kaikki mahdolliset
# parin (X,Y) arvot, kukin tasan kerran.
for (x in 1:6) {
    for (y in 1:6) {
        # T�m�n parin todenn�k�isyys
        pxy = 1/36
        
        # Summan arvo, jos toteutuu t�m� pari
        s = x+y
        
        # Lis�t��n kyseisen summan todenn�k�isyyteen
        f[s] = f[s] + pxy
    }
}

# N�ytet��n arvot
print(f)
\end{lstlisting}

\textbf{Tulostus (2 noppaa):}\\
R tulostaa vektorin (ensimmäinen nolla vastaa mahdotonta summaa 1):
{\tiny
\[
\bigl[\,0,\ 0.02777778,\ 0.05555556,\ 0.08333333,\ 0.11111111,\ 0.13888889,\ 0.16666667,\ 0.13888889,\ 0.11111111,\ 0.08333333,\ 0.05555556,\ 0.02777778\,\bigr]
\]
}

eli
\[
P(S=s)=\frac{1,2,3,4,5,6,5,4,3,2,1}{36}\quad\text{summille }s=2,3,\dots,12.
\]
\pagebreak

\textbf{Muokattu kolmen nopan koodi:}
\begin{lstlisting}[language=R]
# Luodaan vektori f, jonka jokainen komponentti on 0
f = rep(0,18);

# Käydään läpi kaikki mahdolliset
# kolmikot (X,Y,Z), kukin tasan kerran.
for (x in 1:6) {
    for (y in 1:6) {
        for (z in 1:6) {
            # Tämän kolmikoin todennäköisyys
            pxyz = 1/216
            
            # Summan arvo, jos toteutuu tämä kolmikko
            s = x + y + z
            
            # Lisätään kyseisen summan todennäköisyyteen
            f[s] = f[s] + pxyz
        }
    }
}

# Näytetään arvot
print(f)
\end{lstlisting}

\textbf{Pistetodennäköisyydet, 3 noppaa:}

Frekvenssit
\[
1,3,6,10,15,21,25,27,27,25,21,15,10,6,3,1
\]
(=216 yhteensä), joten
\[
P(S=s)=\frac{\text{frekvenssi}(s)}{216},\qquad s=3,\dots,18.
\]
Desimaaleina (pyöristettynä, 2 ekaa mahdotonta jätetty pois):
{\tiny
\[
\begin{array}{c|cccccccccccccccc}
s&3&4&5&6&7&8&9&10&11&12&13&14&15&16&17&18\\\hline
P &.00463&.01389&.02778&.04630&.06944&.09722&.11574&.12500&.12500&.11574&.09722&.06944&.04630&.02778&.01389&.00463
\end{array}
\]
}

\begin{figure}[h]
  \centering
  \includegraphics[width=.3\textwidth]{noppasumma3.png}
%  \caption{ptnf $p_S(s)=P(S=s)$}
\end{figure}


\end{document}