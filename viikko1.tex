\documentclass[12pt,a4paper]{article}
%\documentclass{article}
\input{yhteiset-asetukset.tex}

\title{MAT12003 Todennäköisyyslaskenta I — Viikko 1}
\date{}

\begin{document}

\maketitle

\exercise{1}
Luvuista \textbf{S = \{1, 2, \ldots, 100\}} valitaan umpimähkään yksi. Millä todennäköisyydellä valittu luku on:

\begin{kohta}
    \item \textbf{jaollinen luvulla 13}
    
    Määritellään $A = \{x \in S \mid x \bmod 13 = 0\}$\\
    Eli
    \[
    A = \{13, 26, 39, 52, 65, 78, 91\}, \quad |A| = 7 \text{ ja } |S| = 100
    \]
    \[
    P(A) = \frac{|A|}{|S|} = \frac{7}{100}
    \]

    \item \textbf{kaksinumeroinen, mutta ei jaollinen yhdellätoista} \\
    Määritellään
    \begin{align*}
        B &= \{x \in S \mid 10 \le x \le 99\} \\
        C &= \{x \in B \mid x \bmod 11 = 0\}
    \end{align*}
    \[
    \Rightarrow |B| = 90, \quad C = \{11, 22, \ldots, 99\},\quad |C| = 9
    \]
    Halutaan todennäköisyys, että luku kuuluu $B \setminus C$:
    \[
    P(B \setminus C) = \frac{|B \setminus C|}{|S|} = \frac{90 - 9}{100} = \frac{81}{100}
    \]

    \item \textbf{alkuluku?}

    Määritellään $P = \{x \in S \mid x \text{ on alkuluku}\}$.
    \[
    P = \{2,\ 3,\ 5,\ 7,\ 11,\ 13,\ 17,\ 19,\ 23,\ 29,\ 31,\ 37,\ 41,\ 43,\ 47,\ 53,\ 59,\ 61,\ 67,\ 71,\ 73,\ 79,\ 83,\ 89,\ 97\}
    \]
    \[
    |P| = 25 \quad \Rightarrow \quad P(\text{alkuluku}) = \frac{25}{100} = \frac{1}{4}
    \]
\end{kohta}









\newpage
\exercise{2}

Olkoot A ja B tapahtumia.

\begin{kohta}
  \item \textbf{Lausu joukko-operaatioiden avulla seuraavat tapahtumat: Tapahtumista A ja B}

  \begin{alakohta}
    \item \textbf{sattuu molemmat:}
    
    \[
    A \cap B
    \]

    \item \textbf{ei satu kumpikaan:}
    
    \[
    (A \cup B)^c
    \]

    \item \textbf{sattuu ainakin yksi:}
    
    \[
    A \cup B
    \]

    \item \textbf{sattuu täsmälleen yksi:}
    
    \[
    (A \cup B) \setminus (A \cap B)
    \]
  \end{alakohta}

  \item \textbf{Lausu näiden komplementtitapahtumat sanallisesti.}

  \begin{alakohta}
    \item Molempien tapahtuminen: komplementti on \textbf{"ei tapahdu molemmat"}
    eli tapahtuu korkeintaan toinen tai ei kumpikaan.

    \item Ei kumpikaan: komplementti on \textbf{"ainakin toinen tapahtuu"}.

    \item Ainakin yksi: komplementti on \textbf{"ei tapahdu kumpikaan"}.

    \item Täsmälleen yksi: komplementti on \textbf{"tapahtuu molemmat tai ei kumpikaan"}.
  \end{alakohta}
\end{kohta}










\newpage
\exercise{3}

Kuinka moni joukon $\{1,2,\ldots,20\}$ 4-alkioisista osajoukoista sisältää ainakin jonkin alkion joukosta $\{1, 2, 3, 4, 5\}$?\\

kirjoitetaan:

\begin{kuvaus}
  \item $S$ = $\{1,2,\ldots,20\}$, koko joukko, $|S| = 20$
  \item $A$ = $\{1, 2, 3, 4, 5\}$, joukko kiinnostavia alkioita, $|A| = 5$
  \item $B$ joukon $S$ ne alkiot, jotka \emph{eivät} ole joukossa
  $A$: $B = S \setminus A$, jolloin $|B| = 15$
\end{kuvaus}

Kaikkia nelialkioisia osajoukkoja on:

\[
\binom{20}{4}
\]
\\
Niistä ne, jotka eivät sisällä yhtään alkiota joukosta A,
ovat ne, jotka valitaan pelkästään joukosta B:

\[
\binom{15}{4}
\]
\\
Joten ne osajoukot, jotka sisältävät ainakin jonkin alkion joukosta
A, saadaan vähennyslaskulla:

\[
\binom{20}{4} - \binom{15}{4} = 4845 - 1365 = 3480
\]
\\
Eli, joukon $\{1,2,\ldots,20\}$ 4-alkioisista osajoukoista \textbf{3480} sisältää ainakin jonkin alkion joukosta $\{1, 2, 3, 4, 5\}$


\end{document}