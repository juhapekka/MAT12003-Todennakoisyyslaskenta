\documentclass[12pt,a4paper]{article}
\input{yhteiset-asetukset.tex}

\begin{comment}
\usepackage{pgffor}
\usepackage{xfp}

\foreach \n in {20,...,29} {
  $((41 \cdot \n + 7) \bmod 100) = \the\numexpr (41*\n + 7) \relax \mod 100 = \the\numexpr (41*\n + 7) - 100*((41*\n + 7)/100) \relax$\\
}

\end{comment}

\title{MAT12003 Todennäköisyyslaskenta I — Viikko 2}
\date{}

\begin{document}

\maketitle

\exercise{1}
Olkoot $P(A)=0{,}7$, $P(B)=0{,}3$ ja $P(A\cap B)=0{,}1$.
Määritä seuraavien tapahtumien todennäköisyydet: 

\begin{kohta}
  \item $A\cup B$,
  \[
  P(A\cup B)=P(A)+P(B)-P(A\cap B)=0{,}7+0{,}3-0{,}1=0{,}9
  \]

  \item $B^c$,
  \[
  P(B^c)=1-P(B)=1-0{,}3=0{,}7
  \]

  \item $A^c\cap B^c$,
  \[
  A^c\cap B^c=(A\cup B)^c \;\Rightarrow\; P(A^c\cap B^c)=1-P(A\cup B)=1-0{,}9=0{,}1
  \]

  \item $A^c\cup B$,
  \[
  (A^c\cup B)^c=A\cap B^c \;\Rightarrow\; P(A^c\cup B)=1-P(A\cap B^c)
  \]
  \[
  P(A\cap B^c)=P(A)-P(A\cap B)=0{,}7-0{,}1=0{,}6 \;\Rightarrow\; P(A^c\cup B)=1-0{,}6=0{,}4
  \]

  \item $A\setminus B^c$.
  \[
  A\setminus B^c = A\cap (B^c)^c = A\cap B \;\Rightarrow\; P(A\setminus B^c)=P(A\cap B)=0{,}1
  \]
\end{kohta}








\newpage
\exercise{2}
Olkoon $(\Omega,\mathcal{F},P)$  todennäköisyysavaruus ja $A$ ja $B$ sen tapahtumia. Todista, että

%Olkoon $(\Omega,\mathcal{F},P)$ todennäköisyysavaruus ja $A,B\in\mathcal{F}$. Todista:

\begin{kohta}
  \item \textbf{Jos $A\subseteq B$, niin $P(A)\le P(B)$.}

  Koska $A\subseteq B$, joukko $B$ hajoaa erilliseksi unioniksi $B=A\;\cup\;(B\setminus A)$. 
  täysadditiivisuudesta (TN3):
  \[
  P(B)=P(A)+P(B\setminus A)\ \ge\ P(A)
  \]


    
  \item \textbf{$0\le P(A)\le 1$.}

  Aksioomista $P(A)\ge 0$. Lisäksi $A\subseteq\Omega$, joten yllä olevan todistuksen mukaan
  \[
  P(A)\le P(\Omega)=1
  \]



  \item \textbf{$P(A)\le 1-P(A^c\cap B^c)\le P(A)+P(B)$.}

$(A^c\cap B^c)^c=A\cup B$, joten
  \[
  1-P(A^c\cap B^c)=P\big((A^c\cap B^c)^c\big)=P(A\cup B)
  \]
  Ensimmäinen epäyhtälö: $A\subseteq A\cup B$ \Rightarrow (a): $P(A)\le P(A\cup B)$

  Toinen epäyhtälö: kirjoita erilliseksi yhdisteiksi
  \[
  A\cup B = A\;\cup\;(B\setminus A)
  \]
  jolloin
  \[
  P(A\cup B)=P(A)+P(B\setminus A)\ \le\ P(A)+P(B)
  \]
  Yhdistämällä saadaan
  \[
  P(A)\ \le\ P(A\cup B)\ =\ 1-P(A^c\cap B^c)\ \le\ P(A)+P(B)
  \]
\end{kohta}






\end{document}