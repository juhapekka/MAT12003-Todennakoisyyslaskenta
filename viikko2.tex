\documentclass[12pt,a4paper]{article}
\input{yhteiset-asetukset.tex}

\begin{comment}
\usepackage{pgffor}
\usepackage{xfp}

\foreach \n in {20,...,29} {
  $((41 \cdot \n + 7) \bmod 100) = \the\numexpr (41*\n + 7) \relax \mod 100 = \the\numexpr (41*\n + 7) - 100*((41*\n + 7)/100) \relax$\\
}

\end{comment}

\title{MAT12003 Todennäköisyyslaskenta I — Viikko 2}
\date{}

\begin{document}

\maketitle

\exercise{1}
Olkoot $P(A)=0{,}7$, $P(B)=0{,}3$ ja $P(A\cap B)=0{,}1$.
Määritä seuraavien tapahtumien todennäköisyydet: 

\begin{kohta}
  \item $A\cup B$,
  \[
  P(A\cup B)=P(A)+P(B)-P(A\cap B)=0{,}7+0{,}3-0{,}1=0{,}9
  \]

  \item $B^c$,
  \[
  P(B^c)=1-P(B)=1-0{,}3=0{,}7
  \]

  \item $A^c\cap B^c$,
  \[
  A^c\cap B^c=(A\cup B)^c \;\Rightarrow\; P(A^c\cap B^c)=1-P(A\cup B)=1-0{,}9=0{,}1
  \]

  \item $A^c\cup B$,
  \[
  (A^c\cup B)^c=A\cap B^c \;\Rightarrow\; P(A^c\cup B)=1-P(A\cap B^c)
  \]
  \[
  P(A\cap B^c)=P(A)-P(A\cap B)=0{,}7-0{,}1=0{,}6 \;\Rightarrow\; P(A^c\cup B)=1-0{,}6=0{,}4
  \]

  \item $A\setminus B^c$.
  \[
  A\setminus B^c = A\cap (B^c)^c = A\cap B \;\Rightarrow\; P(A\setminus B^c)=P(A\cap B)=0{,}1
  \]
\end{kohta}


\end{document}