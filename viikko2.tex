\documentclass[12pt,a4paper]{article}
% !TEX program = xelatex
\usepackage[utf8]{inputenc}
\usepackage[T1]{fontenc}
\usepackage[finnish]{babel}
\usepackage[utf8]{inputenc}
\usepackage{graphicx}
\usepackage{titling}
\usepackage{titlesec}
\usepackage{booktabs}
\usepackage{fancyhdr}
\usepackage{lipsum}
\usepackage{comment}
\usepackage{enumitem}
\usepackage{xcolor}
\usepackage{longtable}
%\usepackage{cite}
\usepackage{pgfgantt}
\usepackage{amsmath, amssymb}
\usepackage{tikz}
\usepackage[margin=1in]{geometry}
\usepackage[backend=biber, style=numeric]{biblatex}
%\usepackage{hyperref}
\usepackage{bookmark}
\usepackage{enumitem}
\usepackage{amsmath}
\usepackage{listings}
\lstset{language=Python, basicstyle=\ttfamily\small, breaklines=true,columns=fullflexible}
\lstset{escapeinside={(*@}{@*)}}
\usepackage{fontspec}
\setmainfont{Arial}
\newfontfamily\stylishfont{Noteworthy}
%\newfontfamily\stylishfont{Zapfino}
%\addbibresource{references.bib}
\usetikzlibrary{calc}
\usepackage{xcolor}

\lstdefinestyle{pidstyle}{
    basicstyle=\ttfamily\footnotesize,
    breaklines=true,
    escapechar=\#, % Define escape character for inline LaTeX commands
    linewidth=\textwidth,
    basicstyle=\ttfamily\scriptsize
}

\renewcommand{\maketitle}{%
  \begin{leftmark}
    \vspace*{\baselineskip} % Add a bit of vertical space

%    \includegraphics[width=4cm]{example-image-a} % Add an image before the title. you will need to replace the image path with your own

%    \vspace{0.5cm} % Add vertical space before title

    \textbf{\fontsize{18}{36}\selectfont \thetitle} % Font Size and Bold Title

     \vspace{0.05cm} % Add vertical space before subtitle
%    \textit{\Large \theauthor}  % Subtitle / Author
    \vspace{\baselineskip} % Add vertical space after subtitle
     \rule{\textwidth}{0.4pt} % Add a horizontal line

   \end{leftmark}
%    \thispagestyle{empty} % Prevent header/footer on the title page
}


% Section Formatting
\titleformat{\section}
  {\normalfont\fontsize{18}{22}\bfseries} % Font and style
  {\thesection}         % Section number
  {1em}                   % Horizontal space after section number
  {}                     % Code before the section name
  []                     % Code after the section name

\titleformat{\subsection}
  {\normalfont\fontsize{14}{18}\bfseries} % Font and style
  {\thesubsection}         % Subsection number
  {1em}                   % Horizontal space after subsection number
  {}                     % Code before the subsection name
  []                     % Code after the subsection name

\setlength{\parindent}{0pt}

\title{Computing platforms (Spring 2025)\newline
week 6}
\author{Juha-Pekka Heikkilä}



\pagestyle{fancy}
\fancyhf{}

\renewcommand{\headrulewidth}{0pt}

\newcommand{\footerline}{\makebox[\textwidth]{\hrulefill}}

\newcommand{\footercontent}{%
    \begin{tabular}{@{}l@{}}
        \footerline \\
        \leftmark \hfill \rlap{\thepage}
    \end{tabular}
}

\fancyfoot[C]{\footercontent}


\newcommand{\exercise}[1]{
    \section*{Tehtävä #1}
    \markboth{Tehtävä #1}{}
}

\addtolength{\hoffset}{-1.75cm}
\addtolength{\textwidth}{3.5cm}
%\addtolength{\voffset}{-3cm}
%\addtolength{\textheight}{6cm}
%\setlength{\parindent}{0pt}



% (a), (b), (c)
\newlist{kohta}{enumerate}{1}
\setlist[kohta,1]{
  label=\textbf{\makebox[1cm][l]{\Huge\text{(\stylishfont\alph*)}}},
  leftmargin=!,
  labelindent=0pt
}

% (1), (2), (3)
\newlist{alakohta}{enumerate}{1}
\setlist[alakohta,1]{
  label=\textbf{\makebox[1cm][l]{\Large\text{(\arabic*)}}},
  leftmargin=!,
  labelindent=0pt
}

% termi: selitys
\newlist{kuvaus}{description}{1}
\setlist[kuvaus]{%
  font=\bfseries\stylishfont,
  labelsep=0.5cm,
  leftmargin=2.5cm,
  style=nextline
}

\newcommand{\korostus}[2][yellow]{\colorbox{#1}{\strut #2}}
%\korostus{Yksi kirjoittaja on jo sisällä}
%\korostus[red]{Lukijan täytyy odottaa jos kirjoittajia on paikalla}
%\korostus[orange]{Tämä osa ei ole suoritettavissa}


\newcommand{\evalslantti}[4][-12]{%
%  \left. #2 \,\right|% ei indeksejä tähän
  \mkern-10mu\raisebox{0pt}[0pt][0pt]{\rotatebox{#1}{$\Big|$}}% vinoviiva päälle
  \mkern3mu{}_{\!#3}^{\!#4}% arvot viivan oikealle puolelle
}



\newcommand{\evalraise}{1.2ex}
\newcommand{\evallow}{1.2ex}

% vino eval-viiva, arvot oikealla (oletus: -12)
% \evalslant[asteet]{lauseke}{ala}{yla}
\newcommand{\evalslant}[4][-12]{%
  \left. #2 \,\right.%
  \mkern-10mu\raisebox{0pt}[0pt][0pt]{\rotatebox{#1}{$\Big|$}}%
  \mkern2mu{}^{\raisebox{\evalraise}{$\scriptstyle #4$}}_{\raisebox{-\evallow}{$\scriptstyle #3$}}%
}

\begin{comment}
\usepackage{pgffor}
\usepackage{xfp}

\foreach \n in {20,...,29} {
  $((41 \cdot \n + 7) \bmod 100) = \the\numexpr (41*\n + 7) \relax \mod 100 = \the\numexpr (41*\n + 7) - 100*((41*\n + 7)/100) \relax$\\
}

\end{comment}

\title{MAT12003 Todennäköisyyslaskenta I — Viikko 2}
\date{}

\begin{document}

\maketitle

\exercise{1}
Olkoot $P(A)=0{,}7$, $P(B)=0{,}3$ ja $P(A\cap B)=0{,}1$.
Määritä seuraavien tapahtumien todennäköisyydet: 

\begin{kohta}
  \item $A\cup B$,
  \[
  P(A\cup B)=P(A)+P(B)-P(A\cap B)=0{,}7+0{,}3-0{,}1=0{,}9
  \]

  \item $B^c$,
  \[
  P(B^c)=1-P(B)=1-0{,}3=0{,}7
  \]

  \item $A^c\cap B^c$,
  \[
  A^c\cap B^c=(A\cup B)^c \;\Rightarrow\; P(A^c\cap B^c)=1-P(A\cup B)=1-0{,}9=0{,}1
  \]

  \item $A^c\cup B$,
  \[
  (A^c\cup B)^c=A\cap B^c \;\Rightarrow\; P(A^c\cup B)=1-P(A\cap B^c)
  \]
  \[
  P(A\cap B^c)=P(A)-P(A\cap B)=0{,}7-0{,}1=0{,}6 \;\Rightarrow\; P(A^c\cup B)=1-0{,}6=0{,}4
  \]

  \item $A\setminus B^c$.
  \[
  A\setminus B^c = A\cap (B^c)^c = A\cap B \;\Rightarrow\; P(A\setminus B^c)=P(A\cap B)=0{,}1
  \]
\end{kohta}








\newpage
\exercise{2}
Olkoon $(\Omega,\mathcal{F},P)$  todennäköisyysavaruus ja $A$ ja $B$ sen tapahtumia. Todista, että

%Olkoon $(\Omega,\mathcal{F},P)$ todennäköisyysavaruus ja $A,B\in\mathcal{F}$. Todista:

\begin{kohta}
  \item \textbf{Jos $A\subseteq B$, niin $P(A)\le P(B)$.}

  Koska $A\subseteq B$, joukko $B$ hajoaa erilliseksi unioniksi $B=A\;\cup\;(B\setminus A)$. 
  täysadditiivisuudesta (TN3):
  \[
  P(B)=P(A)+P(B\setminus A)\ \ge\ P(A)
  \]


    
  \item \textbf{$0\le P(A)\le 1$.}

  Aksioomista $P(A)\ge 0$. Lisäksi $A\subseteq\Omega$, joten yllä olevan todistuksen mukaan
  \[
  P(A)\le P(\Omega)=1
  \]



  \item \textbf{$P(A)\le 1-P(A^c\cap B^c)\le P(A)+P(B)$.}

$(A^c\cap B^c)^c=A\cup B$, joten
  \[
  1-P(A^c\cap B^c)=P\big((A^c\cap B^c)^c\big)=P(A\cup B)
  \]
  Ensimmäinen epäyhtälö: $A\subseteq A\cup B$ \Rightarrow (a): $P(A)\le P(A\cup B)$

  Toinen epäyhtälö: kirjoitan erilliseksi yhdisteiksi
  \[
  A\cup B = A\;\cup\;(B\setminus A)
  \]
  jolloin
  \[
  P(A\cup B)=P(A)+P(B\setminus A)\ \le\ P(A)+P(B)
  \]
  Yhdistämällä saadaan
  \[
  P(A)\ \le\ P(A\cup B)\ =\ 1-P(A^c\cap B^c)\ \le\ P(A)+P(B)
  \]
\end{kohta}






\pagebreak
\exercise{3}
Eräässä pelissä käytetään tetraedrin muotoista painotettua noppaa, jolla silmäluvun $k\in\{1,2,3,4\}$ todennäköisyys $p_k$ on kääntäen verrannollinen silmälukuun $k$. Millä todennäköisyydellä nopalla heitetään parillinen silmäluku?
\vspace{0.4cm}



\subsection*{Vaihe 1: Todennäköisyyksien määrittely}
Perusjoukko on $\Omega = \{1, 2, 3, 4\}$.
Tehtävänannon mukaan silmäluvun $k$ todennäköisyys $p_k = P(\{k\})$ on 
kääntäen verrannollinen lukuun $k$ \Rightarrow \ on olemassa 
jokin kerroin $c > 0$, jolle pätee:

\[
p_k = c \cdot \frac{1}{k} \quad \text{kaikille } k \in \Omega.
\]
Määritämme ensin vakion $c$ arvon.

\subsection*{Vaihe 2: $c$:n ratkaiseminen}
Käytämme todennäköisyyden aksioomaa (TN2) $P(\Omega) = 1$.
Koska perusjoukko $\Omega$ on erillisten alkeistapausten 
$\{1\}, \{2\}, \{3\}, \{4\}$ unioni, niiden todennäköisyyksien 
summa on 1:

\[
\sum_{k=1}^{4} p_k = P(\Omega) = 1
\]
Sijoitetaan lauseke $p_k = c/k$ tähän yhtälöön:
\[
p_1 + p_2 + p_3 + p_4 = c \cdot \frac{1}{1} + c \cdot \frac{1}{2} + c \cdot \frac{1}{3} + c \cdot \frac{1}{4} = 1
\]
Ratkaistaan $c$:
\begin{align*}
    c \left( 1 + \frac{1}{2} + \frac{1}{3} + \frac{1}{4} \right) &= 1 \\
    c \left( \frac{12+6+4+3}{12} \right) &= 1 \\
    c \cdot \frac{25}{12} &= 1 \\
    c &= \frac{12}{25}
\end{align*}

Jokaisen alkeistapauksen todennäköisyys on siis $p_k = \frac{12}{25k}$.

\pagebreak
\subsection*{Vaihe 3: Tehtävässä kysytty todennäköisyys}
kiinnostava tapahtuma on $A = \text{saadaan parillinen silmäluku}$.
\[
A = \{2, 4\} = \{2\} \cup \{4\}
\]
Koska tapahtumat $\{2\}$ ja $\{4\}$ ovat toisensa poissulkevia ,
voimme käyttää additiivisuusaksioomaa:
\[
P(A) = P(\{2\} \cup \{4\}) = P(\{2\}) + P(\{4\}) = p_2 + p_4
\]
Sijoitetaan yllä lasketut arvot:
\[
P(A) = \frac{12}{25 \cdot 2} + \frac{12}{25 \cdot 4} = \frac{12}{25} \left( \frac{1}{2} + \frac{1}{4} \right)
\]
Lasketaan lopputulos:
\[
P(A) = \frac{12}{25} \left( \frac{2+1}{4} \right) = \frac{12}{25} \cdot \frac{3}{4} = \frac{36}{100} = \frac{9}{25}
\]

Todennäköisyys saada parillinen silmäluku on $\displaystyle \frac{9}{25}$ eli 0,36







\pagebreak
\exercise{4}
Populaatiossa on 818 henkeä, joista 276 on rokotettu erästä epidemiaa vastaan. Epidemiaan sairastui 69 henkilöä, joista 3 oli rokotettuja.

\begin{kohta}
  \item Mikä on todennäköisyys, että henkilö oli rokotettu ehdolla, että hän sairastui?
  
  Mikä on $P(\text{rokotettu}\mid \text{sairastui})$?

  ehdollisen todennäköisyyden määritelmällä:
  \[
  P(R\mid S)=\frac{P(R\cap S)}{P(S)}
  =\frac{3/818}{69/818}=\frac{3}{69}=\frac{1}{23}\approx 0,0435
  \]
  todennäköisyys, että henkilö oli rokotettu ehdolla, että hän sairastui $\displaystyle \frac{1}{23}\ \approx 0,0435$

  \item Mikä on todennäköisyys, että henkilö sairastui ehdolla, että häntä ei ollut rokotettu?

  Ei-rokotettuja on 818 - 276 = 542 ja heistä sairastui 69 - 3 = 66
  \[
  P(S\mid R^c)
  =\frac{66}{542}=\frac{33}{271}\approx 0{,}1218
  \]
 todennäköisyys, että henkilö sairastui ehdolla, että häntä ei ollut rokotettu $\displaystyle \frac{33}{271}\ \approx 0{,}1218$
\end{kohta}









\pagebreak
\exercise{5}
Laatikossa on 5 punaista ja 7 valkoista palloa.
Kokeessa laatikosta nostetaan 3 palloa ilman takaisinpanoa. Laske
todennäköisyys, että


\begin{kohta}
  \item kaikki pallot ovat punaisia ehdolla, että ainakin yksi on punainen,

  Ehdollinen todennäköisyys:
  \[
    P(\text{3R}\mid \text{väh. 1R})
    = \frac{P(\text{3R})}{P(\text{väh. 1R})}
    = \frac{\binom{5}{3}/\binom{12}{3}}{1 - \binom{7}{3}/\binom{12}{3}}
    = \frac{\binom{5}{3}}{\binom{12}{3}-\binom{7}{3}}
  \]
  Numeroina
  \[
    \frac{\binom{5}{3}}{\binom{12}{3}-\binom{7}{3}}
    = \frac{10}{220-35}
    = \frac{10}{185}
    = \frac{2}{37}
    \approx 0{,}054
  \]

  \item saadaan 1 punainen pallo ja 2 valkoista palloa ehdolla, että kaikki
pallot eivät ole samanvärisiä.

  Ehdollinen todennäköisyys:
  \[
    P(1\text{R}2\text{W}\mid \text{(3R \cup \ 3W)}^C)
    = \frac{P(1\text{R}2\text{W})}{1 - P(\text{3R}) - P(\text{3W})}.
  \]
  Otannalla:
  \[
    P(1\text{R}2\text{W})=\frac{\binom{5}{1}\binom{7}{2}}{\binom{12}{3}},\quad
    P(\text{3R})=\frac{\binom{5}{3}}{\binom{12}{3}},\quad
    P(\text{3W})=\frac{\binom{7}{3}}{\binom{12}{3}}
  \]
  Sijoitus:
  \[
    \frac{\binom{5}{1}\binom{7}{2}/\binom{12}{3}}{1 - \binom{5}{3}/\binom{12}{3} - \binom{7}{3}/\binom{12}{3}}
    = \frac{\binom{5}{1}\binom{7}{2}}{\binom{12}{3} - \binom{5}{3} - \binom{7}{3}}
    = \frac{5\cdot 21}{220 - 10 - 35}
    = \frac{105}{175}
    = \frac{3}{5}
    = 0{,}6
  \]
\end{kohta}








\newpage
\exercise{6}
Juhliin hankitaan 35 MegaMix-karkkipussia, 
40 ÖrkkiKaverit-karkkipussia ja 
15 Salmiakkipläjäys-karkkipussia. 
MegaMixien karkeista 25 \% on salmiakkeja, 
ÖrkkiKaverien karkeista 40 \% on salmiakkeja ja 
Salmiakkipläjäyksessä on vain salmiakkia. 
Yhdestä satunnaisesti valitusta pussista poimitaan 
satunnaisesti yksi karkki.
\vspace{0.4cm}

(Yhteensä $35+40+15 = 90$ pussia)


\begin{kuvaus}
  \item[M] valitaan MegaMix-pussi \Rightarrow   $P(M)=\dfrac{35}{90}$
  \item[O] valitaan ÖrkkiKaverit-pussi \Rightarrow  $P(O)=\dfrac{40}{90}$
  \item[S] valitaan Salmiakkipläjäys-pussi \Rightarrow  $P(S)=\dfrac{15}{90}$
  \item[SK] valittu karkki on salmiakki
\end{kuvaus}

Kussakin pussissa salmiakin ehdollinen todennäköisyys
\[
P(SK\mid M)=0{,}25\qquad
P(SK\mid O)=0{,}40\qquad
P(SK\mid S)=1
\]

\begin{kohta}
  \item Millä todennäköisyydellä poimittu karkki on salmiakki?\\[12pt]
  Kokonaistodennäköisyydellä
  \[
    P(SK)=\!
      \frac{35}{90}\cdot 0{,}25
    + \frac{40}{90}\cdot 0{,}40
    + \frac{15}{90}\cdot 1
    =\frac{53}{120}\approx 0{,}442
  \]

  \item Millä todennäköisyydellä poimittu karkki on peräisin Salmiakkipläjäys-pussista, kun se on salmiakki?\\[12pt]
  Bayesin kaavalla
  \[
    P(S\mid SK)
    =\frac{P(SK\mid S)\,P(S)}{P(SK)}
    =\frac{1\cdot\dfrac{15}{90}}{\dfrac{53}{120}}
    =\frac{20}{53}\approx 0{,}377
  \]
\end{kohta}









\pagebreak
\exercise{7}
Kolmessa lippaassa on kolikot

\[
\begin{array}{ll}
\text{Lipas 1:}& \text{2 kultarahaa (KK)}\\
\text{Lipas 2:}& \text{1 kulta + 1 hopea (KH)}\\
\text{Lipas 3:}& \text{2 hopearahaa (HH)}
\end{array}
\]

Lippaista valitaan ensin umpimähkään yksi ja siitä nostetaan
kolikko umpimähkään ilman takaisinpanoa. Kolikko osoittautuu
kultarahaksi. Mikä on tällä ehdolla todennäköisyys, että samasta lippaasta
nostettu toinenkin kolikko on kultaraha?
\vspace{0.4cm}

$P(1)=P(2)=P(3)=\tfrac13$
\begin{kuvaus}
  \item[L\(_i\)] valittiin lipas \(i\)
  \item[G] ensimmäinen kolikko on kultaa
\end{kuvaus}




\[
P(G\mid L_1)=1\qquad
P(G\mid L_2)=\tfrac12\qquad
P(G\mid L_3)=0
\]
\begin{alakohta}
\item Ensimmäisen kolikon kultatodennäköisyys

\[
P(G)=\sum_{i=1}^{3}P(G\mid L_i)P(L_i)
     =1\cdot\tfrac13+\tfrac12\cdot\tfrac13+0\cdot\tfrac13
     =\tfrac12
\]

\item seuraavat lippaiden todennäköisyydet annetulla kultarahalla

\[
P(L_1\mid G)=\frac{P(G\mid L_1)P(L_1)}{P(G)}
            =\frac{1\cdot\tfrac13}{\tfrac12}=\frac23\qquad
P(L_2\mid G)=\frac{1/2\cdot\tfrac13}{\tfrac12}=\frac13\qquad
P(L_3\mid G)=0
\]

\item toinen kolikko on kultaa



\[
\begin{aligned}
P(\text{2.\ kolikko kultaa}\mid G)
 &= P(L_1\mid G)\,&\cdot&\,P(\text{K}\mid L_1\cap G) &+& P(L_2\mid G)\,&\cdot&\,P(\text{K}\mid L_2\cap G) \\[4pt]
 &= \frac23          &\cdot&\,1 &+& \frac13        &\cdot&\,0
\end{aligned}
\;=\;
\frac23
\]

\vspace{0.4cm}
Eli, todennäköisyys että samasta lippaasta nostettu toinenkin kolikko on kultaraha on
$P= \frac{2}{3}\approx$ 0{,}67

\end{alakohta}









\pagebreak
\exercise{8}
Salama nostaa tavallisesta 52 kortin pakasta kolme korttia.
Hän pitää kaikki nostamansa hertat, laittaa muita maita olevat 
kortit sivuun ja nostaa niiden tilalle jäljellä olevasta pakasta 
uudet kortit. Millä todennäköisyydellä Salamalla on tämän jälkeen 
kädessään kolme herttaa?
\vspace{0.4cm}

Lasketaan aina \textbf{suotuisat jaot / kaikki jaot}
    
\begin{alakohta}
    
\item Jaetaan tapaukset alkuperäisten herttojen lukumäärän mukaan

Olkoon $j\in\{0,1,2,3\}$ = “kuinka monta herttaa tuli ensimmäisellä nostolla”.

\[
P(j)=\frac{\text{(tapoja saada $j$ herttaa ja $3-j$ muuta)}}{\text{(kaikkia kolmen kortin nostoja)}}=
\frac{\binom{13}{j}\binom{39}{3-j}}{\binom{52}{3}}
\]

\item Todennäköisyys täydentää käsi täyteen herttoja

Jos alussa tuli $j$ herttaa, pakkaan jäi $13-j$ herttaa ja kortteja yhteensä $49$.  
Salama nostaa $3-j$ uutta korttia; niiden on oltava kaikki herttoja:


\[
P\!\bigl(\text{onnistuu}\,\mid \,j\bigr)=
\frac{\binom{13-j}{3-j}}{\binom{49}{3-j}}
\qquad \Bigm| \text{kun }j=3\text{ nimittäjä on } \binom{49}{0}=1
\]

\item Kokonaistodennäköisyys

\[
\begin{aligned}
P(\text{3 herttaa lopuksi})
&=\frac{\binom{39}{3}}{\binom{52}{3}}\cdot\frac{\binom{13}{3}}{\binom{49}{3}}
 +\frac{\binom{13}{1}\binom{39}{2}}{\binom{52}{3}}\cdot\frac{\binom{12}{2}}{\binom{49}{2}}
+\frac{\binom{13}{2}\binom{39}{1}}{\binom{52}{3}}\cdot\frac{\binom{11}{1}}{\binom{49}{1}}
 +\frac{\binom{13}{3}}{\binom{52}{3}}\cdot 1 \\[6pt]
&=\frac{9139}{22100}\cdot\frac{286}{18424}
 + \frac{13\cdot 741}{22100}\cdot\frac{66}{1176}
 + \frac{78\cdot 39}{22100}\cdot\frac{11}{49}
 + \frac{286}{22100} \\[4pt]
&=\frac{585\,101}{7\,830\,200}
 \;\approx\;0{,}0747.
\end{aligned}
\]


Salamalla on tämän jälkeen 
kädessään kolme herttaa \approx\, 0.07 todennäköisyydellä.

\end{alakohta}



\pagebreak




\end{document}